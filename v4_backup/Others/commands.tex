%%% Avoid pages jump (comment for final version)

%\let\cleardoublepageold\cleardoublepage
%\let\clearpageold\clearpage

%\renewcommand{\cleardoublepage}{}
%\renewcommand{\clearpage}{}

%%%

\newcommand{\openchapter}{%\pagebreak
\thispagestyle{empty}
\vspace*{1pt}
\pagebreak
}

\newcommand{\includetex}[1]{\openchapter\include{#1}}


%%%

\captionsetup{width=0.95\textwidth}

%%%

\newenvironment{figurethesis}[1]
{
\begin{figure}[htbp]
\begin{center}
}
{
\end{center}
\end{figure}
}
%%%


%%%% Margin %%%%
\newenvironment{changemargin}[2]{\begin{list}{}{%
\setlength{\topsep}{0pt}%
\setlength{\leftmargin}{0pt}%
\setlength{\rightmargin}{0pt}%
\setlength{\listparindent}{\parindent}%
\setlength{\itemindent}{\parindent}%
\setlength{\parsep}{0pt plus 1pt}%
\addtolength{\leftmargin}{#1}%
\addtolength{\rightmargin}{#2}%
}\item }{\end{list}}
%%%% end %%%%

% Mise en forme Tikz (schéma général + logo lupin)
\definecolor{IGNVert}{RGB}{148, 192,  22}
\definecolor{IGNGris}{RGB}{112, 119, 122}
\definecolor{IGNRouge}{RGB}{255, 100, 100}
\definecolor{IGNFonce}{RGB}{55, 58, 60}


\tikzset{
    myArrowIGNGris/.style={->, >=latex,rounded corners, color = IGNGris, thick,font=\scriptsize},
    myArrowDotIGNGris/.style={->, >=latex, densely dashed, shorten >=1pt, rounded corners, color = IGNGris, thick,font=\scriptsize},
    myArrowIGNRouge/.style={-, >=latex, densely dashed, shorten >=1pt, rounded corners, color = IGNRouge, thick,font=\scriptsize},
    myNodeIGNGris/.style={rectangle,rounded corners,draw=black, top color=white, bottom color=IGNGris!80, inner sep=0.5em, minimum size=0.5em, text centered,font=\normalsize },
    myNodeIGNVert/.style={rectangle,rounded corners,draw=black, top color=white, bottom color=IGNVert!80, inner sep=0.5em, minimum size=0.5em, text centered,font=\normalsize },
    myNodeIGNRouge/.style={rectangle,rounded corners,draw=black, top color=white, bottom color=IGNRouge!80, inner sep=0.5em, minimum size=0.5em, text centered,font=\normalsize },
    myNodeIGNRouge1/.style={rectangle,rounded corners, top color=IGNRouge!50, bottom color=IGNRouge, inner sep=0.5em, minimum size=0.5em, text centered,font=\normalsize }
}

%%% Minitoc

\newcommand\ToCrule{\noindent\rule[5pt]{\textwidth}{1.5pt}}

\newcommand\ToCrulev{\tikz\draw[line width=1.5pt, black] (0,0) arc (110:70:1) arc (-110:-70:1) arc (110:70:1) arc (-110:-70:1) arc (110:70:1) arc (-110:-70:1) arc (110:70:1) arc (-110:-70:1) arc (110:70:1) arc (-110:-70:1) arc (110:70:1) arc (-110:-70:1) arc (110:70:1) arc (-110:-70:1)  arc (110:70:1) arc (-110:-70:1) arc (110:70:1) arc (-110:-70:1) arc (110:70:1);}

\newcommand\ToCruleiv{\tikz\draw[line width=1.5pt, black] (0,0) arc (-110:-70:1) arc (110:70:1) arc (-110:-70:1) arc (110:70:1) arc (-110:-70:1) arc (110:70:1) arc (-110:-70:1) arc (110:70:1) arc (-110:-70:1) arc (110:70:1) arc (-110:-70:1) arc (110:70:1) arc (-110:-70:1)  arc (110:70:1) arc (-110:-70:1) arc (110:70:1) arc (-110:-70:1) arc (110:70:1) arc (-110:-70:1);}


\makeatletter
\newcommand\Mprintcontents{%
  \vspace*{-1cm}
  \vfill
  \ToCrulev
  \ttl@printlist[chapters]{toc}{}{1}{}\par\nobreak
  \ToCruleiv
  \vfill
  \pagebreak}
\makeatother


\newcommand{\keyword}[1]{\textbf{#1}}
\newcommand{\tabhead}[1]{\textbf{#1}}
\newcommand{\code}[1]{\texttt{#1}}
\newcommand{\file}[1]{\texttt{\bfseries#1}}
\newcommand{\option}[1]{\texttt{\itshape#1}}

%%% Colors

\definecolor{t1}{RGB}{255, 0, 0 }
\definecolor{t2}{RGB}{0, 255, 0 }
\definecolor{t3}{RGB}{0, 0, 255 }
\definecolor{t4}{RGB}{255, 255, 0 }
\definecolor{t4}{RGB}{200, 200, 0 }
\definecolor{t4b}{RGB}{200, 200, 0 }
\definecolor{t5}{RGB}{255, 127, 0 }
\definecolor{t6}{RGB}{255, 0, 255 }
\definecolor{t7}{RGB}{0, 255, 255 }
\definecolor{t8}{RGB}{200, 0, 100 }
\definecolor{t9}{RGB}{160, 60, 10 }
\definecolor{t10}{RGB}{0, 160, 160 }
\definecolor{t11}{RGB}{135, 135, 0 }
\definecolor{t12}{RGB}{145, 0, 0 }

\definecolor{l0}{RGB}{000, 000, 000}
\definecolor{l1}{RGB}{255, 000, 000}
\definecolor{l2}{RGB}{000, 255, 000}
\definecolor{l3}{RGB}{000, 000, 255}
\definecolor{l4}{RGB}{255, 255, 000}
\definecolor{l5}{RGB}{255, 000, 255}
\definecolor{l6}{RGB}{000, 255, 255}
\definecolor{l7}{RGB}{255, 127, 000}
\definecolor{l8}{RGB}{255, 000, 127}
\definecolor{l9}{RGB}{127, 255, 000}
\definecolor{l10}{RGB}{000, 255, 127}
\definecolor{l11}{RGB}{127, 000, 255}
\definecolor{l12}{RGB}{000, 127, 255}
\definecolor{l13}{RGB}{127, 000, 000}
\definecolor{l14}{RGB}{000, 127, 000}
\definecolor{l15}{RGB}{000, 000, 127}
\definecolor{l16}{RGB}{127, 127, 000}
\definecolor{l17}{RGB}{127, 000, 127}
\definecolor{l18}{RGB}{000, 127, 127}
\definecolor{l19}{RGB}{127, 127, 127}

\definecolor{i0}{RGB}{132, 46, 27}
\definecolor{i1}{RGB}{158, 14, 64}
\definecolor{i2}{RGB}{4, 139, 154}

%%% Colored shapes

\newcommand{\p[1]}{\tikz\draw[#1,fill=#1] (0,0) circle (1.5mm);}
\newcommand{\li[1]}{\tikz\draw[#1,fill=#1] (0,0) -- (0.25,0) -- (0.25,0.25) -- (0,0.25) -- (0,0);}


