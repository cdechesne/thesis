% Chapter 1
\checktoopen
\chapter{State of the art} % Main chapter title
\label{Chapter1} % For referencing the chapter elsewhere, use \ref{Chapter1} 

\startcontents[chapters]
\Mprintcontents


%----------------------------------------------------------------------------------------

% Define some commands to keep the formatting separated from the content

%----------------------------------------------------------------------------------------

\section{Stand segmentation}
One should note that the literature remains focused on individual tree extraction and tree species classification, developing site-specific workflows with similar advantages, drawbacks and classification performance. More authors have focused on forest delineation \citep{eysn2012forest}, that do not convey information about the tree species and their spatial distribution. Consequently, no operational framework embedding the automatic analysis of remote sensing data has been yet proposed in the literature for forest stand segmentation \citep{clement_IJPRS}. \\

In the large amount of literature in the field, only few papers focus on the issue of stand segmentation or delineation. They can be categorized with regard to the type of data processed. \\
First, stand segmentation can be achieved with a single remote sensing source. A stand delineation technique using VHR airborne multispectral imagery is proposed in \citep{leckie2003stand}. The trees are extracted using a valley following approach and classified into 7 tree species (5 coniferous, 1 deciduous, and 1 non-specified) with a maximum likelihood classifier. A semi-automatic iterative clustering procedure is then introduced to generate the forest polygons.\\
A hierarchical and multi-scale approach for the identification of stands is adopted in \citep{hernando2012spatial}. The data inputs were the 4 bands of an airborne 0.5$\:$m orthoimage (Red, Green, Blue, and Near Infra-Red) allowing to derive the Normalized Difference Vegetation Index (NDVI). The stand mapping solution is based on the Object-Based Image Analysis concept. It is composed of two main phases in a cyclic process: first, segmentation, then classification. The first level consists in over-segmenting the area of interest and performing fine-grained land cover classification. The second level aims to transfer the vegetation type provided by a land cover geodatabase in the stand polygons, already retrieved from another segmentation procedure. The multi-scale analysis appears to have a significant benefit on the stand labeling but it is highly heuristic and requires a correct definition of the stand while we consider it is an interleaved problem. \\
A seminal stand mapping method using low density airborne lidar data is proposed in \citep{koch2009airborne}. It is composed of several steps of feature extraction, creation and raster-based classification. Forest stands are created by grouping neighboring cells within each class. Then, only the stands with a pre-defined minimum size are accepted. Neighboring small areas of different forest types that do not reach the minimum size are merged together to an existing forest stand. The approach offers the advantage of detecting 15 forest types that match very well with the ground truth but to the detriment of simplicity: the flowchart has to be highly reconsidered to fit to other stand specifications. Additionally, the tree species discrimination is not addressed.\\
The forest stand delineation proposed in \citep{sullivan2009object} also uses low density airborne lidar still coupling an object-oriented image segmentation and a supervised classification procedure. Three features are computed and rasterized. The segmentation is performed using a region growing approach. Spatially adjacent pixels are grouped into homogeneous discrete image objects or regions. Then, a supervised discrimination of the segmented image is performed using a Battacharya classifier, in order to determine the maturity of the stands. The tree species are ignored and the procedure requires a careful inspection of the raw data both for feature generation and model training. \\
Following the work of \citep{Wulder2008} with IKONOS images, Quickbird-2 panchromatic images are used in \citep{Mora20102474} to automatically delineate forest stands. A standard image segmentation technique is used and the novelty mainly lies on the fact that its initial parameters are optimized with respect to NFI protocols. They show that meaningful stand heights can be derived, which are a critical input for various modeled inventory attributes.\\
The method proposed in \citep{eysn2012forest} aims to generate a forest mask (\textit{forested area} label only) using low density airborne lidar. A Canopy Height Model (CHM) with a spatial resolution of 1$\:$m is derived. The positions and heights of single trees are determined from the CHM using a local maximum filter, based on a moving window approach. Only detected positions with a CHM height superior to 3$\:$m are considered. The crown radii are estimated using an empirical function. The three neighboring trees are connected using a Delaunay triangulation applied to the previously-detected tree position. The crown cover is then calculated using the crown areas of three neighboring trees and the area of their convex hull for each tree triple. The forest mask is derived from the canopy cover values. While this is not a genuine stand delineation method, this approach could be easily extended to a multi-class problem and enlightens the necessity of individual tree extraction even with limited point densities as a basis for the stand-level analysis.\\
A forest stand delineation also based on airborne lidar data is proposed in \citep{wu2014data}. Three features are first directly extracted from the point cloud. A coarse forest stand delineation is then performed on the feature image using the unsupervised Mean-Shift algorithm, in order to obtain under-segmented raw forest stands. A forest mask is then applied to the segmented image in order to retrieve forest and non-forest raw stands. It may create some small isolated areas, iteratively merged to their most similar neighbor until their size is larger than a user-defined threshold in order to product big raw forest stands. They are then refined into finer level using a seeded region growing based on superpixels. The idea is to select several different superpixels in a raw forest stand and merge them. This method provides a coarse-to-fine segmentation with relatively large stands. The process was only applied on a small area of a forest in Finland, thus, general conclusions can not be drawn. \\

Secondly, several methods fusing various types of remote sensing data have also been developed.
The analysis of the lidar and multispectral data is performed at three levels in \citep{tiede2004object}, following a given hierarchical nomenclature of classes in forested environments. The first level represents small objects (single tree scale, individual trees or small groups of trees) that can be differentiated by spectral and structural characteristics using a rule-based classification. The second level corresponds to the stand level. It is built using the same classification process which summarizes forest development phases by referencing to small scale sub-objects at level 1. The third level is generated by merging objects of the same classified forest-development into larger spatial units. The multi-scale analysis offers the advantage of alleviating the standard issue of individual tree crown detection and proposing development stage labels. Nevertheless, the pipeline is highly heuristic, under-exploits lidar data and significant confusion between classes are reported.\\
The automatic segmentation process of forests in \citep{diedershagen2004automatic} is also supplied with lidar and VHR multispectral images. The idea is to divide the forests into higher and lower sections with lidar. An unsupervised classification process is applied to the two new images. The final stand delineation is achieved by segmenting the classification results with pre-defined thresholds. The segmentation results are improved using morphological operators such as opening and closing, which fill the gaps and holes at a specified extent. This method is efficient if the canopy structure is homogeneous and requires a strong knowledge on the area of interest. Since it is based on height information only, it cannot differentiate two stands of similar height but different species.\\
In \citep{leppanen2008automatic} a stand segmentation technique for a forest composed of \textit{Scots Pine}, \textit{Norway Spruce} and \textit{Hardwood} is defined. A hierarchical segmentation on the Crown Height Model followed by a restricted iterative region growing approach is performed on images composed of rasterized lidar data and Colored Infra-Red images. The process was only applied on a limited area of Finland and prevents from drawing strong conclusions. However, the quantitative analysis carried out by the authors shows that lidar data can help to define statistically meaningful stands (here the criterion was the timber volume) and that multispectral images are inevitable inputs for tree species discrimination. \\

\section{Segmentation}
\label{sec:C1_seg}
The direct segmentation of the optical image and/or the lidar point clouds is not sufficient in order to retrieve forest stands. However, with adapted parameters, segmentations algorithms might be useful to obtain relevant over-segmentation of the data \citep{clement_IJPRS}. They can be divide in two categories:
\begin{itemize}
\item The pure segmentation methods, in theses methods, a specific attention must be paid to the choice of the parameter in order to obtain a relevant over-segmentation. Such segmentation can be applied on an image or a point cloud. Specific methods have also been developed for the segmentation of lidar point cloud.
\item The superpixels segmentation methods, they natively produce an over-segmentation of the image. The parameters control the size and the shape of the resulting segments.
\end{itemize}

\subsection{Traditional segmentation methods}
The segmentation of an image can be performed using number of techniques \citep{pal1993review}. \\
The easiest way to segment an image is the thresholding of a gray level histogram of the image \citep{taxt1989segmentation}. When the image is noisy or the background is uneven and illumination is poor, such thresholding might be not sufficient. Thus, adaptive thresholding methods have been developed \citep{yanowitz1989new}. \\

The segmentation can be considered as an unsupervised classification problem. Algorithms dealing with such problems use iterative process. The most popular algorithm is the k-means algorithm. Segmentation methods using the spatial interaction models like Markov Random Field (CRF) \citep{hansen1982image} or Gibbs Random Field (GRF) \citep{derin1987modeling}. Neural networks are also interesting for  image segmentation \citep{ghosh1991image} as they take into account the contextual information. \\

Lastly, the segmentation of an image can be obtained by the detection of the edges of the image \citep{peli1982study}. The idea is to extract points of significant changes in depth values. Edges are local features and are determined based on local information.

\subsection{Segmentation of point cloud}
The segmentation of point cloud has been highly assessed \citep{nguyen20133d}. The aim is to extract meaningful objects. Such extraction has two principal objectives:
\begin{itemize}
\item Objects are detected so as to ease or strengthen subsequent classification task. A precise extraction is not mandatory since the labels would be refined after.
\item Objects are precisely delineated in order to derive features from these objects. A high spatial resolution is therefore expected.
\end{itemize}

In forested areas, the only reliable objects to extract are trees. The first way to extract trees from lidar data is to rasterize the point cloud and use image-based segmentation techniques to obtain trees. Several methods have been developed for single tree delineation \citep{dalponte2014tree, vega2014ptrees, kandare2014new}. 

\subsection{Superpixels methods}
Several superpixels algorithms have been developed \citep{achanta2012slic}. They group pixels into perceptually meaningful atomic regions. Many traditional segmentation algorithms have been employed with more or less success to generate superpixels \citep{shi2000normalized, felzenszwalb2004efficient, comaniciu2002mean, vedaldi2008quick, vincent1991watersheds}. These algorithms produce satisfactory results, however, they may be relatively slow and the number, size and shape of the superpixels might not be specified. \\

Superpixels algorithms have then been developed. One can control the number of superpixel, their size and their shape. \cite{moore2008superpixel} creates superpixels based on a grid. Optimal path are found using graph cut methods. \cite{veksler2010superpixels} proposes a generation of superpixels based on a global optimization. They are obtained by stitching together overlapping image patches such that each pixel belongs to only one of the overlapping regions. \cite{levinshtein2009turbopixels} generate superpixels by a dilatation of a set of seed locations using level-set geometric flow. Resulting superpixels are constrained to have uniform size, compactness, and boundary adherence. Finally, \cite{achanta2012slic} proposes a generation of superpixels based on the k-means algorithms. A weighted distance that combines color and spatial proximity is introduced in order to control the size and the compactness of the superpixels.

\section{Classification}
A classification is a process that aim to categorize observation. The idea is to assign an observation to one or more classes. This can be done manually or algorithmically. The classification can be unsupervised, the classes need to be learned and the observation assigned. Such classification is similar to segmentation (see section \ref{sec:C1_seg}). The classification can be supervised, the target classes are known and observations with labels are available.

\subsection{Supervised classification}

\subsection{Feature selection}


\section{Smoothing methods}

\subsection{Local methods}

\subsection{Global methods}
\stopcontents[chapters]

