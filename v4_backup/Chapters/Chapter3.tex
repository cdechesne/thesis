% Chapter 3

\chapter{Results} % Main chapter title
\label{Chapter3} % For referencing the chapter elsewhere, use \ref{Chapter1} 

\startcontents[chapters]
\Mprintcontents

\section{Data}

\paragraph{VHR optical images \\}
The VHR images are a part of a national database. In this thesis, the images used have a spatial resolution of 50$\:$cm. Two type of ortho-images are available, a color image (3 bands; red: 600-720$\:$nm, green: 490-610$\:$nm and blue: 430-550$\:$nm) and and IRC image (3 bands; near infra-red: 750-950$\:$nm, red and green) captured by the IGN digital cameras \citep{souchon2012large}. It is then possible to obtain four band ortho-images by the combination of the two ortho-images type. \\
\paragraph{Airborne Laser Scanning \\}
IGN also process lots of flights over forested areas with a laser scanning device. The airborne lidar data were collected using an Optech 3100EA device. The footprint was 0.8$\:$m in order to increase the probability to reach the ground. The point density {for all echoes} ranges from 2 to 4$\:$points/m$^{2}$. \\
Data were acquired under leaf-on conditions and fit with the standards used in many countries for large-scale operational forest mapping purposes. \\

A prerequisite for data fusion is the most accurate alignment of the two data \citep{torabzadeh2014fusion}. A frequently used technique is to geo-rectify images using ground controls points (GCPs). A geometric transformation is established between the coordinates of GCPs and their corresponding pixels in the image. It is then applied to each pixel, so that coordinate differences on those points are reduced to the lowest possible level. This method can be easily applied and is relatively fast in terms of computation time. However the use of GCPs can still cause that the unknowns in the trajectory of the platforms produce some remarkable residual errors. Automatic methods for data registration have also been developed \citep{habib2005photogrammetric,mastin2009automatic}. \\

The registration between airborne lidar point clouds and VHR multispectral images was performed by IGN itself using ground control points. This is a standard procedure in the French mapping agency since IGN operates both sensors and has also a strong expertise in data georeferencing (this is in fact the national institute responsible for that in France for both airborne and spaceborne sensors). \\

\paragraph{National Forest Land Cover \\}


\section{Segmentation methods}

\section{Results of the method}
\subsection{Over-segmentation}
\subsection{Feature selection}
\subsection{Classification}
\subsection{Regularization}
%----------------------------------------------------------------------------------------

% Define some commands to keep the formatting separated from the content 
%\newcommand{\keyword}[1]{\textbf{#1}}
%\newcommand{\tabhead}[1]{\textbf{#1}}
%\newcommand{\code}[1]{\texttt{#1}}
%\newcommand{\file}[1]{\texttt{\bfseries#1}}
%\newcommand{\option}[1]{\texttt{\itshape#1}}

%----------------------------------------------------------------------------------------


\stopcontents[chapters]