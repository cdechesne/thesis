% Chapter 3

\chapter{Results} % Main chapter title
\label{Chapter3} % For referencing the chapter elsewhere, use \ref{Chapter1} 

\startcontents[chapters]
\Mprintcontents

\section{Data}

A prerequisite for data fusion is the most accurate alignment of the two data \citep{torabzadeh2014fusion}. A frequently used technique is to geo-rectify images using ground controls points (GCPs). A geometric transformation is established between the coordinates of GCPs and their corresponding pixels in the image. It is then applied to each pixel, so that coordinate differences on those points are reduced to the lowest possible level. This method can be easily applied and is relatively fast in terms of computation time. However the use of GCPs can still cause that the unknowns in the trajectory of the platforms produce some remarkable residual errors. Automatic methods for data registration have also been developed \citep{habib2005photogrammetric,mastin2009automatic}. \\

\section{Segmentation methods}

\section{Results of the method}
\subsection{Over-segmentation}
\subsection{Feature selection}
\subsection{Classification}
\subsection{Regularization}
%----------------------------------------------------------------------------------------

% Define some commands to keep the formatting separated from the content 
%\newcommand{\keyword}[1]{\textbf{#1}}
%\newcommand{\tabhead}[1]{\textbf{#1}}
%\newcommand{\code}[1]{\texttt{#1}}
%\newcommand{\file}[1]{\texttt{\bfseries#1}}
%\newcommand{\option}[1]{\texttt{\itshape#1}}

%----------------------------------------------------------------------------------------


\stopcontents[chapters]