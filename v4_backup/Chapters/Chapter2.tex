% Chapter 2

\chapter{Method} % Main chapter title
\label{Chapter2} % For referencing the chapter elsewhere, use \ref{Chapter1} 

\startcontents[chapters]
\Mprintcontents


%----------------------------------------------------------------------------------------

% Define some commands to keep the formatting separated from the content 
%\newcommand{\keyword}[1]{\textbf{#1}}
%\newcommand{\tabhead}[1]{\textbf{#1}}
%\newcommand{\code}[1]{\texttt{#1}}
%\newcommand{\file}[1]{\texttt{\bfseries#1}}
%\newcommand{\option}[1]{\texttt{\itshape#1}}

%----------------------------------------------------------------------------------------

\section{General flowchart}

With respect to the methods mentioned above, it appears that there are no forest stand segmentation method, based on tree species, that can satisfactorily handle a large number of classes ($>$5). The proposed framework is a fully automatic and modular method for species-based forest stand segmentation. The method is composed of four main steps; over-segmentation feature computation, vegetation type (mainly tree species) classification and regularization (see Figure \ref{fig:flowchart}).

\begin{figure}
\includegraphics[width=\textwidth]{Figures/method.eps}
\caption{Flowchart of the proposed method.}
\label{fig:flowchart}
\end{figure}

Features are first derived at the pixel and at the object level. The most relevant ones are subsequently selected in a supervised way. The objects are extracted using various segmentation methods, since they appear to be sufficient for subsequent steps. A classification is performed at the object level as it significantly improves the discrimination results (about $10\%$ better than the pixel-based approach). This classification is then smoothed. The smoothing may produce homogeneous vegetation type (mainly tree species) areas with smooth borders. The contributions of this method are two-fold:
\begin{itemize}
\item Such framework can be fed with specific constraints allowing to tailor the results to specific criteria (height, age, specie, maturity, density,~\ldots).
\item Here, the training set is automatically derived from an existing forest land-cover geodatabase. Specific attention is paid to the extraction of the most relevant training pixels, which is highly challenging with outdated and generalized vector databases.
\end{itemize}

\section{Over-segmentation}

\subsection{Lidar-based methods}

\subsection{Image-based methods}

\section{Feature extraction}

\section{Classification}

\subsection{Training set design}

\subsection{Feature selection}

\subsection{Classification}

\section{Smoothing}

\subsection{Local methods}

\subsection{Global methods}

\stopcontents[chapters]