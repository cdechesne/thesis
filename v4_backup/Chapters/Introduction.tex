% Introduction

\chapter{Introduction} % Main chapter title
\label{Introduction} % For referencing the chapter elsewhere, use \ref{Chapter1} 

\hyphenation{area}
\hyphenation{areas}

\startcontents[chapters]
\Mprintcontents

%----------------------------------------------------------------------------------------

% Define some commands to keep the formatting separated from the content 

%----------------------------------------------------------------------------------------


\section{Study of forested areas}

Forests are an important component of planet's life. They represent 30\% of the world land surface. They also hold about 90\% of terrestrial biodiversity. Forests are also benefit for the environment; they capt and store the CO$_{2}$ \citep{fahey2010forest}. About 45\% of the total global carbon is held by forests. They also filter dust and microbial pollution of the air \citep{smith2012air}. Finally, They also play an important role in hydrological regulation and water purification \citep{lempriere2008importance}. \\


Forest are complex structures \citep{pommerening2002approaches}, for which informations are needed for management. The information can be the tree species or the tree maturity of the forest. There are two ways to extract such information from forest; field inventory or remote sensing. The field inventories are very expensive to set up and are also not adapted for a national study. A more adapted to obtain such information is remote sensing since it allows to extract them at a large scale. \\

\section{Remote sensing for forested areas}

The analysis of forested areas from a remote sensing point of view can be performed at three different levels: pixel, object (mainly trees) or stand. In statistical national forest inventory (NFI), an automated and accurate tree segmentation is needed in order to extract tree level features (basal area, dominant tree height, etc., \citep{means2000predicting,Malatamo}). However, the tree level is not the only reliable level of analysis for forest studies. When a joint mapping and statistical reasoning is required (e.g., land-cover (LC) mapping and forest inventory), forest stands remain the prevailing scale of analysis \citep{means2000predicting,White2016CJRS}. A stand can be defined in many different ways in terms of homogeneity: tree specie, age, height, maturity, and its definition varies according to the countries. \\

From a remote sensing point of view, the delineation of the stands is a segmentation problem. Forest stands are interesting in order to extract reliable and statistically meaningful features and to provide an input for multi-source statistical inventory. For land-cover mapping, this is highly helpful for forest database updating \citep{Kim09}, whether the labels of interest are \textit{vegetated areas} {(e.g., \textit{deciduous/evergreen/mixed/non-forested)}}, or, more precisely, the tree species. Most of the time in national forestry inventory institutes, for reliability purposes, each area is manually interpreted by human operators with very high resolution (VHR) geospatial images focusing on the infra-red channel \citep{Malatamo}. This work is extremely time consuming and subjective \citep{Wulder2008}. Furthermore, in many countries, the wide variety of tree species (e.g., $>$20) significantly complicates the problem. The design of an automatic procedure based on remote sensing data would fasten such process. Additionally, the standard manual delineation procedure only takes into account the species, and few characteristics (alternatively height, age, stem density or crown closure), while an automatic method could offer more flexibility and would allow to combine characteristics extracted from all complementary data sources. \\

The use of remote sensing data for the automatic analysis of forests has been growing in the last 15 years, especially with the synergistic use of airborne laser scanning (ALS) and optical VHR imagery (multispectral imagery and hyperspectral imagery) \citep{torabzadeh2014fusion,White2016CJRS}. They appear to be both well adapted and complementary inputs for stand segmentation \citep{dalponte2012tree,dalponte2015delineation,7500049}. ALS provides a direct access to the vertical distribution of the trees and to the ground underneath. Hyperspectral and multispectral images are particularly relevant for tree species classification: spectral and textural information from VHR  images can allow a fine discrimination of many species, respectively. Multispectral images are often preferred due to their higher availability, and higher spatial resolution. \\

A prerequisite for data fusion is the most accurate alignment of the two data \citep{torabzadeh2014fusion}. A frequently used technique is to geo-rectify images using ground controls points (GCPS). A geometric transformation is established between the coordinates of GCPs and their corresponding pixels in the image. It is then applied for each pixel, so that coordinate differences on those checkpoints are reduced to the lowest possible level. This method can be easily applied and is relatively fast in terms of computation time. However the use of GCPs can still cause that the unknowns in the trajectory of the platforms produce some remarkable residual errors. Automatic methods for data registration have also been developed \citep{habib2005photogrammetric,mastin2009automatic}. \\

\section{Context of the thesis}
In France, the study of forests is two fold. They need to be mapped and inventoried. The forest inventory allows to obtain the wood stock at a national scale. Statistics such as volume per hectare, deciduous volume or conifer volume can then be derived. The inventory is performed through field inventory and extrapolated using the forest mapping. Thus, the mapping of forest is very important in order to derive accurate statistics. \\
The forest mapping is given by a national forest LC database. It is manually interpreted by human operators with VHR infra-red colored (IRC) ortho-images. It assigns a vegetation type to each mapped beach of more than 5000$\:$m$^{2}$. The nomenclature is composed of 32 classes based on hierarchical criteria such as pure stands of the main tree species of the French forest. The forest LC should be updated in a 10 years cycle.

\section{Objectives}
Currently, the forest LC is obtained through remote sensing (namely photo-inter\-pretation), an method could be developed to update it automatically. Since an old version of the forest LC is available, it can be used as a ground truth input subsequent classification \citep{gressin2013updating}. However, the learning process should be carried out carefully \citep{gressin2014updating}. Indeed, some area might have change (e.g. forest cuts). Furthermore, the database is designed generalized \citep{smith1977database}. A simple classification would then not be sufficient in order to retrieve homogeneous patches similar to the forest LC. Such results could be obtained using smoothing methods \citep{schindler2012overview}. Furthermore, an automatic method would allow to enrich the LC, i.e. retrieve homogeneous tree species stands also homogeneous in terms of height \citep{gressin2014unified}.

\section{Strategy}
Two remote sensing modalities are available for the mapping of forested areas at IGN; VHR optical images and lidar cloud points. The VHR images are a part of a national database. In this thesis, the images used have a spatial resolution of 50$\:$cm. Two type of ortho-images are available, a color image (3 bands; red: 600-720$\:$nm, green: 490-610$\:$nm and blue: 430-550$\:$nm) and and IRC image (3 bands; near infra-red: 750-950$\:$nm, red and green) captured by the IGN digital cameras \citep{souchon2012large}. It is then possible to obtain four band ortho-images by the combination of the two ortho-images type. \\
IGN also process lots of test flight over forested areas with a laser scanning device. The airborne lidar data were collected using an Optech 3100EA device. The footprint was 0.8$\:$m in order to increase the probability to reach the ground. The point density {for all echoes} ranges from 2 to 4$\:$points/m$^{2}$. \\

The registration between airborne lidar point clouds and VHR multispectral images was performed by IGN itself using ground control points. This is a standard procedure in the French mapping agency since IGN operates both sensors and has also a strong expertise in data georeferencing (this is in fact the national institute responsible for that in France for both airborne and spaceborne sensors). \\
Data were acquired under leaf-on conditions and fit with the standards used in many countries for large-scale operational forest mapping purposes. \\

The combination of these two data is very relevant for the study of forest, indeed, optical images provide the major information about the tree species, while lidar give information about the vertical structure of the forest. Furthermore, the lidar allows to extract consistent object such as trees. \\

In order to extract more information from these two modalities, the fusion should be performed at different levels. 3 levels could be defined:
\begin{itemize}
%\addtolength{\itemindent}{3cm}
\item[$\bullet$] Low level: It corresponds to the fusion of the observations, in this case, only the reflectance from the optical images and the height of the lidar points.
\item[$\bullet$] Medium level: It corresponds to the fusion of the features, they are derived at the same level (e.g. the pixel) and merged together. It also corresponds to the cooperative understanding of the data; a feature is derived on a modality (e.g. trees from lidar) and use on the other.
\item[$\bullet$] High level: It corresponds to the fusion of decision. One or many classifications have been performed and the final decision is a smart combination of the classifications and the input data.
\end{itemize}

\section{Structure of the thesis}

\begin{itemize}
\item State of the art: Chapter \ref{Chapter1}
\item Method: Chapter \ref{Chapter2}
\item Results: Chapter \ref{Chapter3}
\item Conclusion and perspectives: Chapter \ref{Conclusion}
\end{itemize}

\stopcontents[chapters]