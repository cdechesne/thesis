% Appendix A

\chapter{Color code} % Main appendix title

\label{AppendixA} % For referencing this appendix elsewhere, use \ref{AppendixA}

\begin{table}
\begin{center}
\begin{tabular}{l l l}
Label & Color & Vegetation type \\
\hline
1 & \p[l1] & {\textit{Chênes décidus}} \\
2 & \p[l2] & {\textit{Chênes sempervirents}} \\
3 & \p[l3] & {\textit{Hêtre}} \\
4 & \p[l4] & {\textit{Châtaignier}} \\
5 & \p[l5] & {\textit{Robinier}} \\
6 & \p[l6] & {\textit{Autre feuillu pur}} \\
7 & \p[l7] & {\textit{Pin maritime}} \\
8 & \p[l8] & {\textit{Pin sylvestre}} \\
9 & \p[l9] & {\textit{Pin laricio ou pin noir}} \\
10 & \p[l10] & {\textit{Pin d'Alep}} \\
11 & \p[l11] & {\textit{Pin à crochet ou pin cembro}} \\
12 & \p[l12] & {\textit{Autre pin pur}} \\
13 & \p[l13] & {\textit{Sapin ou épicéa}} \\
14 & \p[l14] & {\textit{Mélèze}} \\
15 & \p[l15] & {\textit{Douglas}} \\
16 & \p[l16] & {\textit{Autre conifère pur autre que pin}} \\
17 & \p[l17] & {\textit{Lande ligneuse}} \\
18 & \p[l18] & {\textit{Formation herbacée}} \\
19 & \p[l19] & {\textit{Peupleraie}} \\
\end{tabular}
\end{center}
\caption{Vegetation type color code}
\end{table}