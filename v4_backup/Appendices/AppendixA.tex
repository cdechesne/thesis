% Appendix A

\chapter{Color code} % Main appendix title

\label{AppendixA} % For referencing this appendix elsewhere, use \ref{AppendixA}

\begin{table}
\begin{center}
\begin{tabular}{l l}
\p[l1] & {\textit{Chênes décidus}} \\
\p[l2] & {\textit{Chênes sempervirents}} \\
\p[l3] & {\textit{Hêtre}} \\
\p[l4] & {\textit{Châtaignier}} \\
\p[l5] & {\textit{Robinier}} \\
\p[l6] & {\textit{Autre feuillu pur}} \\
\p[l7] & {\textit{Pin maritime}} \\
\p[l8] & {\textit{Pin sylvestre}} \\
\p[l9] & {\textit{Pin laricio ou pin noir}} \\
\p[l10] & {\textit{Pin d'Alep}} \\
\p[l11] & {\textit{Pin à crochet ou pin cembro}} \\
\p[l12] & {\textit{Autre pin pur}} \\
\p[l13] & {\textit{Sapin ou épicéa}} \\
\p[l14] & {\textit{Mélèze}} \\
\p[l15] & {\textit{Douglas}} \\
\p[l16] & {\textit{Autre conifère pur autre que pin}} \\
\p[l17] & {\textit{Lande ligneuse}} \\
\p[l18] & {\textit{Formation herbacée}} \\
\p[l19] & {\textit{Peupleraie}} \\
\end{tabular}
\end{center}
\caption{Vegetation type color code}
\end{table}